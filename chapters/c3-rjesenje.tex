\chapter{Prijedlog rješenja}
\paragraph*{}
U skladu sa datim zahtjevima predložena je izrada aplikacijske platforme pod nazivom Logit (LAPP), opisane u nastavku, sa detaljnim tehničkim detaljima u narednim poglavljima. Uzimajući u obzir data ograničenja, te funkcionalne i nefunkcionalne zahtjeve određeno je da se korisnička aplikacija izradi na Android platformi sa podrškom za Android API nivo počevši od nivoa 19 (4.4 KitKat), to je najniži nivo koji omogućava korištenja naprednih NFC i kriptografskih funkcionalnosti te osigurava dobru pokrivenost potencijalne korisničke baze sa ukupnom adopcijom od preko 90\% za navedenu ili višu verziju\cite{droidstats} podržavajući uređaje unazad četiri godine. Za uspješan rad aplikacije neophodno je da korisnički uređaj podržava i NFC funkcionalnosti, prema prognozama analitičke kuće IHS Technology, do 2020. godine svaki treći uređaj imati će podršku za NFC.\cite{nfcforecast}
\paragraph*{}
Uvodi se dodatno pojam lokacijskog dokaza\cite{locproof} koji u širem smislu u kontekstu podređenog korisnika (en. slave), obuhvata kriptografski potpisan korisnički identitet, korisnički uređaj, vrijeme i GPS lokacijske podatke korisničkog uređaja. Za svrhu osiguranja jedinstvenosti identiteta i vjerodostojnosti potvrde lokacijskih dokaza odabrano je korištenje RSA asimetrične enkripcije, gdje se pri uspješnoj autentifikaciji generiše jedinstveni set ključeva za korisnički uređaj, privatnom dijelu ključa nije moguće pristupiti izvan aplikacije (SEC1), niti je moguće eksportovati ključ (SEC2), a u određenom vremenskom period može postojati samo jedan valjan set ključeva za jednog korisnika jer se raniji ključevi ne uzimaju u obzir ukoliko postoji noviji set (SEC3), sprječavajući tako replikaciju identiteta na više uređaja.
\paragraph*{}
Pored Android komponente aplikacije izrađena je i serverska aplikacija u programskom jeziku Python (LAPI), čija je namjena posredovanje u komunikaciji sa autentifikacijskim agentom (ZAMGER), te pohranjivanja i održavanje javnih korisničkih kriptografskih ključeva (CERT) i njihovo povezivanje sa autentifikacijskim podacima korisnika, pored toga služi i kao repozitorij za potpisana prisustva (ATTN). Na ovu komponentu se može gledati kao na integrisani namjenski repozitorij korisničkih certifikata i domenski repozitorij neporecivih i neizmjenjivih lokacijskih dokaza (SPIM).
\paragraph*{}
Budući da na Elektrotehničkom fakultetu u Sarajevu postoji SSO (en. Single-Sign On) politika autentifikacije, u serverskoj komponenti (LAPI) je implementiran autentifikacijski posrednik koji prilikom prvog pokretanja aplikacije prijavljuje korisnika koristeći postojeće pristupne podatke, tom prilikom u slučaju uspješne prijave generiše se i jedinstveni set RSA ključeva dužine 2048 bita (KEYS), koji se pohranjuju na korisničkom uređaju (DEVICE), a javni dio, tj. certifikat (CERT) se pohranjuje i u repozitorij ključeva (LAPI) sa poveznicom na korisnički identitet, kasnije se ti certifikati koriste za provjeru valjanosti potpisa lokacijskih dokaza (SPIM).

\paragraph*{}
Da bi se osigurala jednostavnost korištenja aplikacije odabrana je implementacija HCE emulacijskog načina rada NFC komunikatora koji omogućava korisniku da izvrši komunikaciju sa drugim uređajem bez potrebe da pokreće aplikacijski prozor na svom uređaju, dovoljno je da upali ekran svoj uređaja i prinese ga master (M) uređaju koji prikuplja potpise, u ovom slučaju drugoj instanci Logit aplikacije na kojoj je pokrenuta aktivnost za prikupljanje potpisa (LAPP).

\paragraph*{}
Približavanjem mobilnih uređaja (BUMP) otvara se jednosmjerni komunikacijski kanal u smijeru od slave (S) prema master (M) uređaju korištenjem ISO/IEC 14443 Tip A komunikacijskog protokola pri čemu se emulira NFC Forum Tag tipa 4 i putem NDEF Aplikacije prenosi jedna NDEF poruka (NDEFMSG) koja sadrži vremensko-lokacijski dokaz potpisan od strane korisnika, nadalje u tekstu takav objekat nazivati ćemo SPIM (en. spime)\cite{bruces}.

\paragraph*{}
Po primitku poruke nadređeni uređaj (en. master) koji osluškuje da mu se pridruže podređeni uređaji (en. slave) i ima pokrenutu Logit aplikaciju, tu poruku sprema u lokalni repozitorij potpisa ukoliko ona zadovolja uslove da očitana slave GPS lokacija nije udaljena više od 50 metara od očitane master GPS lokacije (VK1 - validacijski kriterij \#1), te da podešena razlika satova master i slave uređaja nije veća od 300 sekundi (VK2), bez da nad SPIM objektom vrši ikakve izmjene, ukoliko SPIM objekat ne zadovoljava date validacijske kriterije odbija se i ispisuje se odgovarajuća poruka na master ekranu. Moguće je naknadno klikom na validacijsko dugme (ACTVAL) u korisničkom interfejsu izvršiti provjeru svih prikupljenih potpisa tokom jedne sesije (SESS), tom prilikom se, ukoliko postoji mrežna veza; svi potpisi pošalju Logit serveru (LAPI) na provjeru i vraća se stanje valjanosti potpisa za sve proslijeđene SPIM objekte.

\paragraph*{}
Ukoliko master (M) želi da pohrani SPIM objekte iz jedne sesije (SESS) na Logit server (LAPI), klikom na sinhronizacijsko dugme u interfejsu (ACTSYNC), on vrši dodatno potpisivanje svakog SPIM objekta svojom komponentom privatnog ključa (MPRK), tako što potpiše hash (SHA256) vrijednost SPIM objekta (AID) sa dodatim svojim jedinstvenim master korisničkim imenom (MUSER) i jedinstvenim identifikatorom sesije (SID) i dodatno generiše SHA256 vrijednosti tih potvrda (CID), nakon čega objedinjuje sve CID vrijednosti i dodatno ih potpisuje svojim MPRK, sve te vrijednosti šalje Logit server (LAPI) na pohranjivanje, ovakvom procedurom se obezbjeđuje neporecivost i neizmjenjivost SPIM i SESS objekata, jer onemogućava izmjene pojedinačnih SPIM objekata, te brisanje ili dodavanje objekata u finaliziranoj sesiju (SESS) od strane malicioznih aktera bez da naruši integritet SHA256 vrijednosti.

\paragraph*{}
Uzmimajući u obzir bitnost rješenja i visoku vjerovatnoću svakodnevne primjene kod ciljane korisničke grupe, te izazove koje takav slučaj korištenja predstvalja omogućena je i direktna e-mail komunikacija za prijavu grešaka ili slanje prijedloga sa glavnog korisničkog interfejsa (ACTBUG).