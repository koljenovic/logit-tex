\chapter{Zaključak}
Sa aspekta teorije igara\cite{davis2012game} problem dokaza u okruženju dva igrača unutar kolaborativne igre\cite{nash1953two} je uvijek prost i sa te strane ne postoji vektor napada, pa ne zavređuje poseban sigurnosni tretman. Stoga ukoliko se prikupljanje obostrano korisnih dokaza u vidu prisustva predavanju svede u navedene okvire tada se radi o kolaborativnoj igri dva igrača i ne bi trebao da se javlja problem povjerenja (\textit{en. trust}), za ovu klasu problema specifični su problemi raspodjele, ali oni dolaze do izražaja tek u igrama sa više igrača i rješavaju se tako što se svaki doprinos svakog učesnika valorizira relativno prema već postojećoj vrijednosti i kao takav bespovratno dodaje u dijeljeni registar (\textit{en. ledger})\cite{antonopoulos2014mastering}.

\paragraph*{}
Aplikacija izrađena u okviru ovog rada zadovoljava preduvjete kolaborativne igre \textbf{dva igrača}, budući da je prisustvo studenata dokaz prisustva profesora i obratno, njihov odnos je kriptografski osiguran hijerarhiji nalik na hijerarhiju uređenog hash skupa. Dodamo li u igru trećeg igrača, npr. instituciju univerziteta, situacija se pretvara u naizgled mnogo kompleksniju igru nepoznatog tipa sa <N> igrača, no to ovdje nije slučaj jer je svaku igru sa <N> igrača moguće predstaviti kao graf <N> igara DVA igrača, stim što se nameće zahtjev da cijeli taj graf moraja biti poznat svim igračima u svakom trenutku (potpuna cache koherentnost), odmah je uočljivo da bi to bilo problematično za ostvariti u realnim uslovima, uistinu to jeste tako, ali ukoliko se opusti samo jedan kriterij, npr. onaj savršene latencije - nazire se postojanja rješenja, takvo rješenje postoji u vidu PBFT algoritma\cite{lamport1982byzantine} i unaprijeđeno je kroz brojne varijante Paxos rješenja\cite{lamport2001paxos}.

\paragraph*{}
Možemo zaključiti da ukoliko bi se u okviru jedne institucije implementiralo kompletno rješenje kolaborativne igre sa <N> igrača, zadovoljavajuće propusnosti \textit{(en. bandwidth)} za datu namjenu; data institucija bi zadovoljila neophodne kriterije za digitalizaciju svih veza povjerenja, npr. u slučaju univerziteta - fakultet, studijski program, index, diploma, potvrda, naučni rad, ispit etc. Navedeni koncepti bi kriptografski mogli praviti povjerljive i istinite kolaborativne relacije između sebe ili sa drugim konceptima unutar skupa objekata, stoga uključujući navedenu relaciju povjerenja slobodni smo govoriti o objektnom modelu povjerenja u digitalnom obliku.

\paragraph*{}
Kompletna struktura navedenog modela polazi, počiva i završava na pretpostavci o povjerljivosti i jedinstvenosti \textbf{korisničkog privatnog ključa}. Predstavljanjem ostalih vrsta igara (npr. nekolaborativne zero-sum) sa <N> igrača u obliku grafa <N> igara sa DVA igrača može se po potrebi kompletirati bilo kakav model povjerenja ili nepovjerenja, samim time i diskutovati o inžinjeringu povjerenja u određenom kontekstu.