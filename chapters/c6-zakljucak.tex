\chapter{Zaključak}
Aplikacija izrađena u okviru ovog rada zadovoljava zahtjeve navedene u postavci zadatka i pripadajućem opisu. Korištene su savremene kriptografske metode za implementaciju sigurnosno osjetljivih funkcionalnosti i osigurano je stabilno okruženje za neometano funkcionisanje aplikacije, dodatno je prema zahtjevima uspješno realiziran NFC komunikacijski interfejs između studentskih i instruktorskih mobilnih uređaja. U cilju lakšeg skaliranja težilo se je što više koristiti standardizovane tehnologije, posebno kada je u pitanju NFC, gdje je dodatno implementirana emulacija NTAG vrste taga kao NDEF medija, time je omogućeno da se sistem u budućnosti prilagodi stacionarnim NFC čitačima i korištenju samostalnih NFC tagova.

\paragraph*{}
Pokušana je pilot primjena sistema u saradnji sa nastavnim osobljem na predmetu "Tehnologije sigurnosti" u školskoj godini 2017/18. kojom prilikom je sačinjen spisak studenata i izvršene pripreme sistema. Navedena pilot primjena okončana je neuspješno zbog otvorenih sigurnosnih pitanja u integraciji sa postojećim sistemima, nedostatka resursa i nepostojanja pokusnog sistema pogodnog za projekte u ranoj fazi testiranja, stoga u cilju povećanja inovativnosti i razvoja novih usluga preporučuje se izrada pokusnih \textit{(en. staging)} sistema odvojenih od produkcijskog u okviru Elektrotehničkog fakulteta u Sarajevu.

\paragraph*{}
Tokom pripreme pilot primjene identifikovano je da značajan broj studenata ne posjeduje NFC omogućene mobilne uređaje, te su za njihove potrebe izrađene NTAG216 NFC token naljepnice, no primjena navedenih tokena uvjetovana je dodatnim istraživanjem i doradom Logit sistema za rad sa NTAG216 da bi osigurao isti ili viši nivo sigurnosnih garancija od onog koje pruža Android izvršno okruženje. Kao dodatna smjernica u istraživanju dat je prijedlog korištenja QR kodova za namjenu supstitucije u slučajevima nepostojanja tehničkih predispozicija za upotrebu sistema na strani korisnika, navedena tehnologija može dati dobre rezultate u praktičnoj primjeni i zavređuje dalji istraživački tretman.

\paragraph*{}
Krajnja težnja Logit rješenja je obuhvatanje cjelokupnog sistema autentifikacije i modeliranje relacija povjerenja u materijalnopravnom okruženju, kroz izradu proširive bazne platforme koja može obuhvatiti digitalizaciju mnoštva svakodnevnih administrativnih zadataka jedne institucije, sa tim ciljem daljnje istraživačke napore zavređuje usmjeriti ka razvoju stabilne PKI infrastrukture, kao i digitalizaciji vjerodostojnih institucionalnih registara poput registra ispita sa ciljem digitalizacije studentskog indeksa i srodnih dokumenata.