\chapter{Postavka problema} \label{chapter:problem}
Projektni zadatak ovog završnog rada je izrada aplikacije na Android platformi sa pripadajućom udaljenom serverskom komponentom, koje u cjelini treba da omoguće evidentiranje prisustva nastavnim aktivnostima na Elektrotehničkom fakultetu u Sarajevu. U skladu sa zadatim funkcionalnim zahtjevima, a iz razloga olakšanog korištenja i praktičnosti upotrebe neophodno je iskoristiti beskontaktne komunikacijske mogućnosti savremenih mobilnih telefona u vidu NFC komunikacijskog protokola.

\paragraph*{}
Također neophodno je osigurati korisnike aplikacije od mogućih zloupotreba korištenjem dostupnih kriptografskih metoda i tehnologija, te stvoriti neophodne uslove za sticanje povjerenja u širi sistem bilježenja prisustva putem neporecivosti i neizmjenjivosti prethodno unesenih podataka. Poželjna mogućnost je jednostavna integracija sa postojećim sistemima, prvenstveno onim autentifikacijskim i autorizacijskim, te planiranje arhitekture za buduća proširenja u vidu omogućavanja integracije sa infrastrukturnim hardverskim čitačima i TAG karticama.

\paragraph*{}
Potrebno je dokumentovati proces izrade i opisati korištene tehnologije, sa posebnim osvrtom na korištene kriptografske metode i tehnologije, te identifikovati otvorena pitanja na polju elektronskih registara prisustva, mogućnosti i izazove koje oni predstavljaju uz rješenja koja navedena aplikacija nudi u datom kontekstu.

\paragraph*{}
Prvobitna motivacija prilikom odabira istraživačkog problema pogodnog za završni rad master studija bila je za nešto opširniji problem potpune digitalizacije dokumenta studentskog indeksa, čije rješenje bi zasigurno imalo višestruke koristi, no naknadnim konsultacijama je ustanovljeno kako bi dati problem zahtijevao rješenje velikog obima i kao takav nije pogodan za obradu u okviru jednog završnog rada, za kompromisno rješenje odabrana je ponuđena tema "Aplikacija za evidentiranje prisustva" specificirana u okviru prethodno opisane postavke problema, budući da dijeli mnoštvo funkcionalnih i infrastrukturnih zahtjeva sa prvobitnom temom, a bez naknadne administrativne kompleksnosti. Stoga se digitalizacijom prikupljanja prisustva pored olakšanog odvijanja nastave nastoji otvoriti put za implementaciju elektronske studijske dokumentacije, sa ciljem unaprjeđenja obrazovnog procesa i institucije univerziteta.