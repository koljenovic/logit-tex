\chapter{Sigurnosna analiza}
Sa aspekta teorije igara\cite{davis2012game} u okruženju dva igrača unutar kooperativne igre\cite{nash1953two} pronalazak optimalne strategije je uvijek prost u slučaju da međusobna saradnja maksimizira profit oba igrača, stoga ukoliko se prikupljanje obostrano korisnih dokaza u vidu prisustva predavanju svede u navedene okvire tada se radi o kolaborativnoj igri studenta i predavača te ne bi trebao da se javlja problem međusobnog povjerenja, dovoljan uvjet za sigurnost takvog sistema je ispravna tehnička implementacija kriptografskih rješenja koja sprječavaju napade jednostranim falsifikovanjem dokaza. Aplikacija izrađena u okviru ovog rada zadovoljava preduvjete kolaborativne igre \textbf{dva igrača}, budući da je prisustvo studenata dokaz prisustva profesora i obratno, njihov odnos je kriptografski osiguran strukturom nalik na hijerarhiju uređenog hash skupa, takva struktura osigurava osobine neizmjenjivosti i neporecivosti.

\paragraph*{}
Sa tehničkog aspekta jedan vektor napada predstavlja lažiranje lokacije ili vremena na korisničkim uređajima, stim što je preduvjet za uspješnost takvog napada koluzija predavača i studenta da priskrbe korist na štetu treće strane, tj. institucije korisnice sistema i neučesnika u koluziji. Moguće je otežati izvodivost i osigurati detekciju ovog napada programskim mjerama zabrane i bilježenja korištenja ručno podešene (\textit{en. mock}) lokacije uređaja i provjerom vremenskih podataka potpisa na LAPI strani koja osigurava pouzdane vremenske podatke. Dodatno treba napomenuti da cijena opisanog napada raste proporcionalno broju studenata učesnika jer svi studenti koji žele priskrbiti neostvarenu korist moraju učestvovati u koluziji, također svi neučesnici imaju štetu (npr. predavanje nije održano u predviđenom terminu zbog odsustva predavača, naknadno falsifikovan dokaz o održavanju u koluziji sa dva studenta, svi ostali studenti gube bodove za prisustvo), stoga i direktu korist od razotkrivanja napada.

\paragraph*{}
Sigurnost izloženog rješenja polazi i počiva na pretpostavci povjerljivosti i jedinstvenosti \textbf{korisničkog privatnog ključa} koji nikada ne napušta okruženje emuliranog sigurnosnog elementa korisničkog Android uređaja, ukoliko se naruši data pretpostavka ne postoje garancije sigurnosti sistema. Postoji jedan-na-jedan asocijacija između korisnika i sigurnosnog uređaja, gdje se zbog prirode problema i niskog nivoa prijetnje kao dovoljan uvjet asocijacije uzima posjedovanje uređaja, no takav uvjet ne može se smatrati dovoljnim za čvrst dokaz identiteta te se za namjene gdje je to neophodno preporučuje implementacija dodatnog faktora biometrijske identifikacije unutar pouzdanog okruženja (ne na korisničkom uređaju).