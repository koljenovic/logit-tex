\chapter{Uvod}
Prodor digitalnih računara i komunikacijskih tehnologija u sve sfere ljudskog života i djelovanja, te dramatično povećanje broja korisnika interneta u posljednjoj deceniji nametnulo je mnoštvo novih društvenih i tehničkih izazova. Društveni izazovi najbolje su uočljivi kroz višedecenijsku debatu o privatnosti i vlasništvu nad ličnim podacima, samim time zadiru duboko u diskusiju o ljudskim pravima i identitetu sa jedne i često suprostavljenim komercijalnim interesima sa druge strane. Ukoliko se u tom kontekstu posmatra aktuelna EU uredba o zaštiti podataka\cite{gdpr} (\textit{en. GDPR}) postaje jasno da su digitalna tehnologija i komunikacije postale integralni dio društvene i emocionalno-psihološke realnosti\cite{Searle1995}, do te mjere da se digitalni tragovi smatraju dijelom nepovredivog identiteta osobe. Iz navedenog je jasno da se radi o institucionalizaciji jedne potpuno nove društveno-tehnološke paradigme unutar pravnih okvira Europske unije.

\paragraph*{}
Sa tehničke strane, dostignuća na poljima kriptografije, teorije mreža i novih komunikacijskih tehnologija, te njihova široka prihvaćenost otvorila su mogućnosti izrade računarskih sistema spremnih da odgovore na novonastale društvene izazove u okviru opisane nove paradigme. Ovakvi novonastali računarski sistemi kao dodatno izvršno okruženje imaju društveno-pravnu realnost te se u tim okvirima izvršavaju masovno, dobrovoljno, distribuirano i interaktivno\cite{Cahill2003} van centralizovanog računarskog izvršnog okruženja u smislu Von Neumannove arhitekture. Opisani sistemi mogu se okarakterisati kao sistemi potpomaganja (\textit{en. assist}), npr. kriptografski računarski sistem u domenu autentifikacije i autorizacija u novoj paradigmi postmatra se kao sistem računarski-potpomognutog povjerenja, ekvivalentno višem nivou apstrakcije.

\paragraph*{}
Registri u kontekstu društvenih institucija su elementarni mehanizam sistema povjerenja, sigurnosne karakteristike takvih institucionalnih registara stoga čine osnov istraživačkog interesa u domenu institucionalne sigurnosi. Napredni elektronski registri izrađeni korištenjem kriptografskih tehnika i savremenih komunikacijskih protokola za prikupljanje i obradu podataka omogućavaju poboljšanje njihovih sigurnosnih osobina, otvarajući nove načine primjene i stvarajući uslove za viši nivo društvenog razvoja i institucionalne efikasnosti, uz to pružaju i adekvatan odgovor na novonastale društvene izazove. Stoga, ukoliko se obezbijede i ispoštuju preduslovi izrade sigurnog sistema\cite{iso2013iso}, evidenciju prisustva u vidu naprednog elektronskog registara treba posmatrati i kao vremensko-prostorni dokaz određenog događaja, ovaj rad usmjeren je na izradu jednog takvog sistema računarski-potpomognutog povjerenja u obliku institucionalnog registra elektonske evidencije prisustva.