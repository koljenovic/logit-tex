\chapter{Uvod} \label{chapter:intro}
Prodor digitalnih računara i komunikacijskih tehnologija u sve sfere ljudskog života i djelovanja, te dramatično povećanje broja korisnika interneta u posljednjoj deceniji nametnulo je mnoštvo novih društvenih i tehničkih izazova. Društveni izazovi najbolje su uočljivi kroz višedecenijsku debatu o privatnosti i vlasništvu nad ličnim podacima, samim time zadiru duboko u diskusiju o ljudskim pravima i identitetu sa jedne i često suprostavljenim komercijalnim interesima sa druge strane. Ukoliko se u tom kontekstu posmatra aktuelna EU uredba o zaštiti podataka\cite{gdpr} (\textit{en. GDPR}) postaje jasno da su digitalna tehnologija i komunikacije postale integralni dio društvene i emocionalno-psihološke realnosti\cite{Searle1995}, do te mjere da se digitalni tragovi smatraju dijelom nepovredivog identiteta osobe. Iz navedenog je jasno da se radi o institucionalizaciji jedne potpuno nove društveno-tehnološke paradigme unutar pravnih okvira Europske unije.

\paragraph*{}
Sa tehničke strane, dostignuća na poljima kriptografije, teorije mreža i novih komunikacijskih tehnologija, te njihova široka prihvaćenost otvorila su mogućnosti izrade računarskih sistema spremnih da odgovore na novonastale društvene izazove u okviru opisane nove paradigme. Ovakvi novonastali računarski sistemi kao dodatno izvršno okruženje imaju društveno-pravnu realnost te se u tim okvirima izvršavaju masovno, dobrovoljno, distribuirano i interaktivno\cite{Cahill2003} van centralizovanog računarskog izvršnog okruženja u smislu Von Neumannove arhitekture. Opisani sistemi mogu se okarakterisati kao sistemi potpomaganja (\textit{en. assist}), npr. kriptografski računarski sistem u domenu autentifikacije i autorizacija u novoj paradigmi postmatra se kao sistem računarski-potpomognutog povjerenja, ekvivalentno višem nivou apstrakcije.

\paragraph*{}
Registri u kontekstu društvenih institucija su elementarni mehanizam sistema povjerenja, sigurnosne karakteristike takvih institucionalnih registara stoga čine osnov istraživačkog interesa u domenu institucionalne sigurnosi. Napredni elektronski registri izrađeni korištenjem kriptografskih tehnika i savremenih komunikacijskih protokola za prikupljanje i obradu podataka omogućavaju poboljšanje njihovih sigurnosnih osobina, otvarajući nove načine primjene i stvarajući uslove za viši nivo društvenog razvoja i institucionalne efikasnosti, uz to pružaju i adekvatan odgovor na novonastale društvene izazove. Stoga, ukoliko se obezbijede i ispoštuju preduslovi izrade sigurnog sistema\cite{iso2013iso}, evidenciju prisustva u vidu naprednog elektronskog registara treba posmatrati i kao vremensko-prostorni dokaz određenog događaja, ovaj rad usmjeren je na izradu jednog takvog sistema računarski-potpomognutog povjerenja u obliku institucionalnog registra elektonske evidencije prisustva.

\paragraph*{}
Krajnji cilj je izrada korisničke aplikacije koja omogućava praktično bilježenje prisustva studenata predavanjima u okviru univerzitetskog okruženja, korištenjem Android platforme i NFC komunikacijskog protokola, te uspostava osnovne infrastrukture registra prisustva, prvenstveno u demonstrativne svrhe, ali i kao polazne tačke za dalja proširenja infrastrukture javnog ključa.

\paragraph*{}
Korištenje Android platforme, zbog njene široke rasprostranjenosti treba da omogući što veću dostupnost rješenja krajnjim korisnicima, sa mogućnošću implementacije dopunskih tehnologija u slučaju da korisnik ne posjeduje Android mobilni uređaj. NFC tehnologija omogućava brzu i praktičnu komunikaciju korisničkih uređaja bez potrebe za uspostavom i održavanjem komunikacijskog kanala, dovoljan uvjet za ostvarenje komunikacije je staviti uređaje u neposrednu blizinu, što pogoduje slučajevima upotrebe gdje se fizička blizina podrazumijeva kao jedan od bitnih zahtjeva. Posebnost izrađenog rješenja je i u tome da postoje dvije odvojene grupe korisnika, nastavno osoblje - koje prikuplja potpise i studenti - koji prisustvo prijavljuju, te obje grupe po potrebi istu aplikaciju mogu koristiti i za namjenu prikupljanja, kao i prijave prisustva.

\paragraph*{}
Pisani rad pored predstavljanja teoretskih osnova rješenja ima i dodatnu ulogu funkcionalne i tehničke dokumentacije, pa je i sam sadržaj dat tim redoslijedom. Poglavlje \ref{chapter:problem} precizira nešto detaljniju postavku samog problema koji rad tretira kao i osnovne motivacije koje su dovele do izrade ovakvog rješenja, u poglavlju \ref{chapter:crypto} detaljno su obrađene kriptografske osnove rješenja koje omogućavaju sigurnosne garancije neophodne za ispravan rad. Prijedlog rješenja u vidu logičkog i tehničkog modela dat je u poglavlju \ref{chapter:solution}. Razumijevanje opisanog konceptualnog modela aplikacije, pored korištenih tehnologija u poglavlju \ref{chapter:tech} neophodno je da bi se moglo diskutovati o mnogobrojnim detaljima same implementacije u poglavlju \ref{chapter:implementation}. Poglavlje \ref{chapter:sec} vrši kratku ali obuhvatnu sigurnosnu analizu rješenja sa zaključkog datim u poglavlju \ref{chapter:conclusion}. Preporučeno je rad studiozno čitati od početka do kraja, sa povremenim referisanjem na ranija poglavlja po potrebi.

\paragraph*{}
Detaljna postavka projektnog zadatka sa eksplicitno navedenim tehnologijama i smjernicama za izradu u najvećoj mjeri je doprinijela i samom metodološkom pristupu istraživačke faze razvoja aplikacijskog rješenja, ona se prvenstveno ogledala u pronalasku i sintezi raznovrsnih komunikacijskih i kriptografskih standarda koji propisuju detalje implementacije neophodne za razvoj zadatog projektnog rješenja i međusobnu interoperaciju u okviru Android platforme. Iako je polazna literatura bila vrlo korisna za razumijevanje funkcionalnog modela NFC komunikacije, za samu implementaciju najkorisnijom se je pokazala službena dokumentacija uz nadopunu relevantnim standardima.