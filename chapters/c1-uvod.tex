\chapter{Uvod}
Dostignuća na poljima kriptografije i digitalnih komunikacijskih tehnologija, te njihova široka prihvaćenost otvorila su mogućnosti za izradu računarskih sistema potpomognutog povjerenja, ranije domenu isključivo ljudskog interesovanja u obliku društvenih sistema ograničenih konceptom reputacije i pouzdanosti pojedinaca. Savremeno društvo u velikoj mjeri formirano je upravo na te dvije temeljne vrijednosti i mnoge društvene pojedinosti, običaji i procesi su u osnovi mehanizmi zaštite i očuvanja tih vrijednosti, stoga je od ključne važnosti iskorititi mogućnosti digitalnih tehnologija za unapređenje postojećih sistema povjerenja.

\paragraph*{}
Registri u kontekstu društvenih institucija jedan su od oblika i mehanizama takvih sistema povjerenja i karakteristika pouzdanosti ima ključnu ulogu za njihov takav status. Dodatno elektronski registri u kombinaciji sa korištenjem kriptografskih metoda provjere i osiguravanja, te digitalnih komunikacijskih protokola za prikupljanje i obradu podataka omogućavaju značajno poboljšanje njihove osnovne svrhe i otvaraju nove mogućnosti njihove primjene, stvarajući uslove za viši nivo društvenog razvoja i institucionalne efikasnosti. Stoga, ukoliko se obezbijede i ispoštuju preduslovi izrade sigurnog sistema, elektronsku evidenciju prisustva u kontekstu digitalnih registara treba posmatrati i kao vremensko-prostorni dokaz određenog događaja, ovaj rad usmjeren je na izradu jednog takvog institucionalnog registra elektonske evidencije prisustva.